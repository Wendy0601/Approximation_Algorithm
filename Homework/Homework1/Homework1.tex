\documentclass[a4paper,12pt]{article}
%改变页边距
\usepackage[english]{babel}
\usepackage{geometry}
\geometry{left=1.5cm,right=1.5cm,top=2.0cm,bottom=2cm}
\usepackage{latexsym}
\usepackage{amsmath}
\usepackage{times}
\usepackage{graphicx}
\usepackage{epstopdf}
\usepackage{booktabs}
\usepackage{float}
\usepackage[numbers]{natbib}
\citestyle{IEEE}
\usepackage{subfigure}
\usepackage{indentfirst} % 段首空格
%\usepackage[utf8]{inputenc}
%\usepackage[english]{babel}
\usepackage{gensymb}
\newtheorem{theorem}{Theorem}[section]
\newtheorem{corollary}{Corollary}[theorem]
\newtheorem{lemma}[theorem]{Lemma}
\usepackage{ gensymb }
\usepackage{indentfirst} 
\usepackage{times}
\begin{document}
	\title{Homework 1}
	\author{
			Wenting LI
		\and
			liw14@rpi.edu
			}
	\maketitle


 \section*{1. }
\subsection*{a. }
Given a $(\Delta+1)$-colorable graph, where $\Delta$ is the maximum degree of any vertex in $G=(V,E)$,

\begin{enumerate}
\item[1. ]Find the vertex $v$ in $G$ with the maximum degree $\Delta$ and its neighbor $N(v)=\{w | (v,w) \in E, i=1,\dots,\Delta \}$;

\item[2. ]First color $v$ with one color, then color its neighbors $w_i \in N(v)$ one by one. If $w_i$ is also connected to $w_j, i\neq j$, then color them with different colors. In this way, we need at most $\Delta +1$ colors to color $v$ and its neighbor  $N(v)$;

\item[3. ]Then find another $v^{\prime}$ not colored yet and its neighbor $N(v^{\prime})=\{w | (v^{\prime},w) \in E, i\leq \Delta \}$. Color them in the same way as step (2) using the $\Delta +1$ colors;

\item[4. ]Continue to color all other vertexes with less than $\Delta +1$ colors until the graph $G$ is completely colored.
\end{enumerate}

As we can color G in poly-time with at most $(\Delta +1)$ colors, then $G$ is $(\Delta +1)$ colorable.
\subsection*{b. }
Bipartite graph is an undirected graph with nodes partitioned into group $X$ and $Y$. For each line, its one ending node is in group $X$ and the other ending node is in $Y$. 

Then we can color all the nodes in group $X$ with one color and the nodes in group $Y$ with the other color. Therefore, bipartite graphs are $2$-colorable. 

\subsection*{c.}
Given undirected 3-colorable graph $G=(V,E)$,  $|V|=n$.
For each $v \in V$ vertext, let $d(v)$ be the degree of $v$.

If $d(v)\geq \sqrt{(n)}$, then remove $v$ and its neighbors $N(v)=\{w_i| (v,w_i) \in E, i\geq \sqrt{(n)}\}$, and color $v$ with the first color and then color its neighbor $w_1$ with the second color, and then color all other neighbors one by one. If two of its neighbors $w_i, w_j$ are connected, then color them with the second and the third color, otherwise color the neighbors with the second color. We can color the subgraph with a poly-time $O(n)$.

As graph $G$ is a 3 colorable graph and thus its subgraph is also a 3-colorable graph, thus we need at most 3 colors to color each subgraph. 

Each time we remove at most $\sqrt{n}+1$ vetex from $G$, which has $n$ vertex in total, thus we need to remove at most $ \sqrt{(n)} $. As we need color each subgraph with at most 3 colors, then in order to color all these vertex and its neighbors, we need at most $O(\sqrt{n})$ colors. 

Until all the vertex in $G$ have degree less than $\sqrt{(n)}$, we can apply the algorithm in (a) to color them with $O(\sqrt{(n)})$ colors with a poly-time.

In summary, we need $O(\sqrt{n}) $ colors.
\subsection*{d.}
Given undirected 3-colorable graph $G=(V,E)$,  $|V|=n$.
For each $v \in V$ vertext, let $d(v)$ be the degree of $v$.\\
If $d(v)\geq n^{\frac{2}{3}}$, then remove $v$ and its neighbors $N(v)=\{w_i| (v,w_i) \in E, i\geq \sqrt{(n)}\}$, and color $v$ with the first color and then color its neighbor $w_1$ with the second color, and then color all other neighbors one by one. If two of its neighbors $w_i, w_j$ are connected, then color them with the second and the third color, otherwise color the neighbors with the second color. We can color the subgraph with a poly-time $O(n)$.

As graph $G$ is a 4 colorable graph and thus its subgraph is also a 3-colorable graph, thus we need at most 4 colors to color each subgraph. 

Each time we remove at most $(n^{\frac{2}{3}}+1)$ vetex from $G$, which has $n$ vertex in total, thus we need to remove at most $ n^{\frac{1}{3}} $. As we need color each subgraph with at most 4 colors, then in order to color all these vertex and its neighbors, we need at most $O(n^{\frac{1}{3}})$ colors. 

Until all the vertex in $G$ have degree less than $O(n^{\frac{1}{3}})+O(n^{\frac{2}{3}})$, we can apply the algorithm in (a) to color them with $O(n^{\frac{2}{3}})$ colors with a poly-time.

In summary, we need $O(n^{\frac{2}{3}}) $ colors.

\subsection*{e. }
\noindent \textbf{Claim:}
Suppose VERTEX COLORING is approximable to a factor better than $\frac{4}{3} k$ for a $k$-colorable graph when $P \neq NP$, then determining if a graph is 3-colorable is solvable in poly-time.

$\textbf{Proof:} \Rightarrow $  

If VERTEX COLORING is approximable to a factor better than $\frac{4}{3} k$ when $P \neq NP$, then we can find a poly-time algorithm that using less $\frac{4}{3} k$ colors for a $k$-colorable graph. 

When $k=3$ and $\frac{4}{3} k=4$, as $ 3 < \frac{4}{3} k$, then we can find a poly-time algorithm to determine if a graph is 3-colorable.

This is contradict to the fact that determining if a graph is 3-colorable is NP-Complete. Therefore, VERTEX COLORING  is not approximable to a factor better than $\frac{4}{3}$ unless $P=NP$.
 
 
 
\end{document}	
